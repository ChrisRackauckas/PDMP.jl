\documentclass[article]{jss}

%%%%%%%%%%%%%%%%%%%%%%%%%%%%%%
%% declarations for jss.cls %%%%%%%%%%%%%%%%%%%%%%%%%%%%%%%%%%%%%%%%%%
%%%%%%%%%%%%%%%%%%%%%%%%%%%%%%

%% almost as usual
\author{Romain Veltz\\INRIA \And
        Simon Frost\\University of Cambridge}
\title{Piecewise Deterministic Markov Processes in Julia: \pkg{PDMP.jl}}

%% for pretty printing and a nice hypersummary also set:
\Plainauthor{Romain Veltz, Simon Frost} %% comma-separated
\Plaintitle{Piecewise Deterministic Markov Processes in Julia: PDMP.jl} %% without formatting
\Shorttitle{\pkg{PDMP.jl}: Piecewise Deterministic Markov Processes} %% a short title (if necessary)

%% an abstract and keywords
\Abstract{
  Simulation of stochastic processes is an important part of modeling and inference in many biological fields, including neuroscience, epidemiology population genetics, and systems biology. We present \pkg{PDMP.jl}, a package written in \proglang{Julia} for the efficient simulation of stochastic processes that comprise of both a discrete and a continuous component.}

\Keywords{Markov processes, simulation, \proglang{Julia}}
\Plainkeywords{Markov processes, Julia} %% without formatting
%% at least one keyword must be supplied

%% publication information
%% NOTE: Typically, this can be left commented and will be filled out by the technical editor
%% \Volume{50}
%% \Issue{9}
%% \Month{June}
%% \Year{2012}
%% \Submitdate{2012-06-04}
%% \Acceptdate{2012-06-04}

%% The address of (at least) one author should be given
%% in the following format:
\Address{
  Romain Veltz\\
  MathNeuro\\
  Institut National de Recherche en Informatique et en Automatique\\
  E-mail: \email{romain.veltz@inria.fr}\\
  URL: \url{https://romainveltz.pythonanywhere.com/}
}
%% It is also possible to add a telephone and fax number
%% before the e-mail in the following format:
%% Telephone: +43/512/507-7103
%% Fax: +43/512/507-2851

%% for those who use Sweave please include the following line (with % symbols):
%% need no \usepackage{Sweave.sty}

%% end of declarations %%%%%%%%%%%%%%%%%%%%%%%%%%%%%%%%%%%%%%%%%%%%%%%


\begin{document}

%% include your article here, just as usual
%% Note that you should use the \pkg{}, \proglang{} and \code{} commands.

% \section[About Java]{About \proglang{Java}}
%% Note: If there is markup in \(sub)section, then it has to be escape as above.

\section[Introduction]{Introduction}

The True Jump Method (TJM, \cite{Veltz2015}) is an algorithm for the simulation of PDMPs.

\section[The PDMP.jl package]{The \pkg{PDMP.jl} package}

The \pkg{PDMP.jl} package is written in the \proglang{Julia} programming language.

The True Jump Method involves solving stiff ordinary differential equations. We use the CVODE routine available in the \pkg{Sundials.jl} package; in preliminary analyses, CVODE was approximately twice as fast as the ode23s routine, implemented in native \proglang{Julia} code in the \pkg{ODE.jl} package.

\section[Examples]{Examples}

The interface to the simulations closely resembles that used in the \proglang{R} package \pkg{GillespieSSA} \citep{Pineda-Krch2008}.

\subsection[The Morris-Lecar model]{The Morris-Lecar model}

\begin{Code}
using PDMP, JSON, GR

const p0  = convert(Dict{AbstractString,Float64}, JSON.parsefile("ml.json")["type II"])
const p1  = ( JSON.parsefile("ml.json"))
include("morris_lecar_variables.jl")
const p_ml = ml(p0)

function F_ml(xcdot::Vector{Float64}, xc::Vector{Float64},xd::Array{Int64},t::Float64, parms::Vector)
  # vector field used for the continuous variable
  #compute the current, v = xc[1]
  xcdot[1] = xd[2] / p_ml.N * (p_ml.g_Na * (p_ml.v_Na - xc[1])) + 
    xd[4] / p_ml.M * (p_ml.g_K  * (p_ml.v_K  - xc[1]))  +
    (p_ml.g_L  * (p_ml.v_L  - xc[1])) + p_ml.I_app
  nothing
end

function R_ml(xc::Vector{Float64},xd::Array{Int64},t::Float64, parms::Vector, sum_rate::Bool)
  if sum_rate==false
    return vec([p_ml.beta_na * exp(4.0 * p_ml.gamma_na * xc[1] + 4.0 * p_ml.k_na) * xd[1],
                p_ml.beta_na * xd[2],
                p_ml.beta_k * exp(p_ml.gamma_k * xc[1] + p_ml.k_k) * xd[3],
                p_ml.beta_k * exp(-p_ml.gamma_k * xc[1]  -p_ml.k_k) * xd[4]])
  else
    return (p_ml.beta_na * exp(4.0 * p_ml.gamma_na * xc[1] + 4.0 * p_ml.k_na) * xd[1] +
              p_ml.beta_na * xd[2] +
              p_ml.beta_k * exp( p_ml.gamma_k * xc[1] + p_ml.k_k) * xd[3] +
              p_ml.beta_k * exp(-p_ml.gamma_k * xc[1] - p_ml.k_k) * xd[4])
  end
end

function Delta_ml(xc::Array{Float64},xd::Array{Int64},t::Float64,parms::Vector,ind_reaction::Int64)
  # this function return the jump in the continuous component
  return true
end

immutable F_type; end
call(::Type{F_type},xcd, xc, xd, t, parms) = F_ml(xcd, xc, xd, t, parms)

immutable R_type; end
call(::Type{R_type},xc, xd, t, parms, sr) = R_ml(xc, xd, t, parms, sr)

immutable DX_type; end
call(::Type{DX_type},xc, xd, t, parms, ind_reaction) = Delta_ml(xc, xd, t, parms, ind_reaction)

xc0 = vec([p1["v(0)"]])
xd0 = vec([Int(p0["N"]),    #Na closed
           0,               #Na opened
           Int(p0["M"]),    #K closed
           0])              #K opened

nu = [[-1 1 0 0];[1 -1 0 1];[0 0 -1 1];[0 0 1 -1]]
parms = vec([0.])
tf = p1["t_end"]

dummy_t = chv(6,xc0,xd0, F_ml, R_ml,(x,y,t,pr,id)->vec([0.]), nu , parms,0.0,0.01,false)
srand(123)
dummy_t = @time chv(4500,xc0,xd0, F_ml, R_ml,(x,y,t,pr,id)->vec([0.]), nu , parms,0.0,tf,false)
result =  PDMP.chv_optim(2,xc0,xd0,F_type,R_type,DX_type,nu,parms,0.0,tf,false)
srand(123)
result =  @time PDMP.chv_optim(4500,xc0,xd0,F_type,R_type,DX_type,nu,parms,0.0,tf,false) #cpp= 100ms/2200 jumps
println("#jumps = ", length(dummy_t.time)," ", length(result.time))
try
  println(norm(dummy_t.time-result.time))
  println("--> xc_f-xc_t = ",norm(dummy_t.xc-result.xc))
  println("--> xd_f-xd_t = ",norm(dummy_t.xd-result.xd))
end
GR.plot(result.time,result.xc[1,:],"y",result.time, 0*result.xd[3,:],title = string("#Jumps = ",length(result.time)))
\end{Code}

\section[Future directions]{Future directions}

\section[Acknowledgements]{Acknowledgements}

\bibliography{pdmp}

\end{document}
